\documentclass[titlepage,11pt]{scrartcl}
\usepackage{graphicx}
\usepackage[utf8]{inputenc}
\usepackage{amsmath}
\usepackage{amsmath}
\usepackage{amsfonts}
\usepackage{amssymb}
\usepackage{listings}
\usepackage[pdftex]{hyperref}
\usepackage[x11names,table]{xcolor}
\usepackage{graphicx}
\usepackage{float}

\title{	
    \normalfont\normalsize
	\vspace{25pt}
	{\huge Simulación\ Proyecto 1}
	\vspace{12pt}
}

\author{\LARGE Enrique Martínez González C-412}

\date{}

\begin{document}

\maketitle

\section{Problema:}

	\textbf{Puerto Sobrecargado (Overloaded Harbor)}

	En un puerto de supertanqueros que cuenta con 3 muelles y un remolcador para la descarga de estos barcos de manera simultánea se desea conocer el tiempo promedio de espera de los barcos para ser cargados en el puerto. El puerto cuenta con un bote remolcador disponible para asistir a los tanqueros. Los tanqueros de cualquier tamaño necesitan de un remolcador para aproximarse al muelle desde el puerto y para dejar el muelle de vuelta al puerto. El tiempo de intervalo de arribo de cada barco distribuye mediante una función exponencial con $\lambda = 8$ horas. Existen tres tamaños distintos de tanqueros: pequeño, mediano y grande, la probabilidad correspondiente al tamaño de cada tanquero se describe en la tabla siguiente. El tiempo de carga de cada tanquero depende de su tamaño y los parámetros de distribución normal que lo representa también se describen en la tabla siguiente.

	\begin{center}
		\begin{tabular}
			{c c c}
			\rule[-1ex]{0pt}{1.5ex} Tamaño & Probabilidad de Arribo & Tiempo de Carga \\
			\rule[-1ex]{0pt}{1.5ex} Pequeño & 0.25 & $\mu = 9, \sigma^2 = 1$ \\
			\rule[-1ex]{0pt}{1.5ex} Mediano & 0.25 & $\mu = 12, \sigma^2 = 2$ \\
			\rule[-1ex]{0pt}{1.5ex} Grande & 0.5 & $\mu = 18, \sigma^2 = 3$ \\
		\end{tabular}
	\end{center}

	De manera general, cuando un tanquero llega al puerto, espera en una cola (virtual) hasta que exista un muelle vacío y que un remolcador esté disponible para atenderle. Cuando el remolcador está disponible lo asiste para que pueda comenzar su carga, este proceso demora un tiempo que distribuye exponencial con $\lambda = 2$ horas. El proceso de carga comienza inmediatamente después de que el barco llega al muelle. Una vez terminado este proceso es necesaria la asistencia del remolcador (esperando hasta que esté disponible) para llevarlo de vuelta al puerto, el tiempo de esta operación distribuye de manera exponencial con $\lambda = 1$ hora. El traslado entre el puerto y un muelle por el remolcador sin tanquero distribuye exponencial con $\lambda = 15$ minutos. Cuando el remolcador termina la operación de aproximar un tanquero al muelle, entonces lleva al puerto al primer barco que esperaba por salir, en caso de que no exista barco por salir y algún muelle esté vacío, entonces el remolcador se dirige hacia el puerto para llevar al primer barco en espera hacia el muelle vacío; en caso de que no espere ningún barco, entonces el remolcador esperará por algún barco en un muelle para llevarlo al puerto. Cuando el remolcador termina la operación de llevar algún barco al puerto, este inmediatamente lleva al primer barco esperando hacia el muelle vacío. En caso de que no haya barcos en los muelles, ni barcos en espera para ir al muelle, entonces el remolcador se queda en el puerto esperando por algún barco para llevar a un muelle. Simule completamente el funcionamiento del puerto. Determine el tiempo promedio de espera en los muelles.

\section{Principales Ideas seguidas para la solución del problema:}

\section{Modelo de Simulación de Eventos Discretos desarrollado para resolver el
problema:}

\section{Consideraciones obtenidas a partir de la ejecución de las simulaciones del
problema:}

\section{Enlace al repositorio del proyecto en Github:}
	\url{https://github.com/kikeXD/simulation_project_1}
\end{document}